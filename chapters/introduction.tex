\chapter{INTRODUCTION}
\label{chapter:introduction}

Contactless payments also known as Near Field Communication \cite{want2011near} (NFC) payments is getting more 
and more popular throughout the world. In 2019, the global NFC market size was valued at \$15 million and the 
projection shows that by 2028 it will reach \$54 million \cite{nfcmarket}. Moreover, the COVID-19 pandemic has 
significantly increased the use of contactless payments \cite{contactlesscovid}. 
To make a contactless payment, one needs to own a contactless-enabled card which can be used near a NFC Point of 
Sale (POS). Also, major phone vendors like Google, Samsung, Apple has started manufacturing NFC-enabled smart 
devices which can be used to make a contactless payment. Digital NFC-wallet applications can be installed and 
activated with a contactless credit/debit card and then used to make a transaction near a contactless-supported 
terminal.  

Since Android 4.4 introduced Host based card emulation \cite{hostbased} which allows an Android application to 
emulate a card and communicate directly with an NFC reader, bypassing the necessity of a Secure Element (SE), 
developers can now build digital NFC-wallet extending Android's Host Apdu Service class\cite{hostapdu}. Among the 
available NFC-wallets, Google Pay\cite{gpay} and Apple Pay are the most prominent wallet applications being used 
widely. In addition to these third party applications, many banks develop their own digital wallet application for 
their customers. To build such an application, the developers follow the standard protocol described by 
EMV (Europay Mastercard Visa)\cite{van2016emv}. 

Recent works have shown that EMV contactless payment protocol \cite{emvco} has vulnerabilities that can be 
exploited to make an unauthorized purchase \cite{emms2016contactless}\cite{francis2009potential}. 
\citet{emvandvulnerabilities} presented an overview of the EMV protocol and how it's vulnerable to different 
types of security attacks. However, the focus of previous research work in this field is on the protocol itself, 
rather than the implementation of the protocol. We are interested in verifying the protocol implementation to 
ensure that the EMV specifications are properly followed by NFC-wallet applications' developers.

PayFuzz aims to find the gap between the EMV protocol specification and it's implementation on real-life wallet 
applications. We feed the wallet payment interface with our constructed commands and record the response to 
create a finite state machine(FSM) using LearnLib similar to \cite{formalmodels}. Discrepancies between state 
machines from different wallet applications move us to the evaluation stage where we find vulnerabilities in 
the implementation which can be leveraged to launch an attack on the wallet application. Specifically, the goal is to 
develop rigorous techniques for formally studying the security and compliance issues of payment applications 
(wallets and POS) with automated reasoning (i.e., formal methods and program analysis) and build defenses. We also 
aim to create formal specifications and enable secure code generation for critical components as a long-term goal.



In this work, to extract a behavioral abstraction from NFC wallets, we use an active automata learning approach \cite{dikeue} 
\cite{dynamictesting}. 
