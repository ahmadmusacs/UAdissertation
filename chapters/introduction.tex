\chapter{INTRODUCTION}
\label{chapter:introduction}

Contactless payments also known as Near Field Communication \cite{want2011near} (NFC) payments is getting more 
and more popular throughout the world. In 2019, the global NFC market size was valued at \$15 million and the 
projection shows that by 2028 it will reach \$54 million \cite{nfcmarket}. Moreover, the COVID-19 pandemic has 
significantly increased the use of contactless payments \cite{contactlesscovid}. 
To make a contactless payment, one needs to own a contactless-enabled card which can be used near a NFC Point of 
Sale (POS). Also, major phone vendors like Google, Samsung, Apple has started manufacturing NFC-enabled smart 
devices which can be used to make a contactless payment. Digital NFC-wallet applications can be installed and 
activated with a contactless credit/debit card and then used to make a transaction near a contactless-supported 
terminal.  

Since Android 4.4 introduced Host based card emulation \cite{hostbased} which allows an Android application to 
emulate a card and communicate directly with an NFC reader, bypassing the necessity of a Secure Element (SE), 
developers can now build digital NFC-wallet extending Android's Host Apdu Service class\cite{hostapdu}. Among the 
available NFC-wallets, Google Pay\cite{gpay} and Apple Pay are the most prominent wallet applications being used 
widely. In addition to these third party applications, many banks develop their own digital wallet application for 
their customers. To build such an application, the developers follow the standard protocol described by 
EMV (Europay Mastercard Visa)\cite{van2016emv}. 

Recent works have shown that EMV contactless payment protocol \cite{emvco} has vulnerabilities that can be 
exploited to make an unauthorized purchase \cite{emms2016contactless}\cite{francis2009potential}. 
\citet{emvandvulnerabilities} presented an overview of the EMV protocol and how it's vulnerable to different 
types of security attacks. However, the focus of previous research work in this field is on the protocol itself, 
rather than the implementation of the protocol. We are interested in verifying the protocol implementation to 
ensure that the EMV specifications are properly followed by NFC-wallet applications' developers.

\textbf{Goal: }We aim to find the gap between the EMV protocol specification and it's implementation on real-life wallet 
applications. We feed the wallet payment interface with our constructed input messages and record the responses to 
create a finite state machine(FSM) following a regular inference approach similar to \cite{formalmodels}. The FSMs are then checked for EMV specification 
compliance. We plan to perform i) differential analysis of multiple implementations through equivalence 
checks \cite{7491755}, ii) state machine verification of security properties \cite{Hussain2018LTEInspectorAS}, 
and iii) protocol verification under Dolev-Yao adversary \cite{securityofpublickey} 
between wallet and POS \cite{10.1007/978-3-642-39799-8_48}. Futhermore, we plan to extend our analysis to Point of 
Sale (POS) machines. The final analysis involves both wallet and POS acting in a two party system. 
Specifically, the goal is to develop rigorous techniques for formally studying the security and 
compliance issues of payment applications (wallets and POS) with automated reasoning (i.e., formal methods and program analysis) and build defenses. We also 
aim to create formal specifications and enable secure code generation for critical components as a long-term goal.

\textbf{Challenges: }There are several challenges to enable automated reasoning and physical layer defense.
\begin{itemize}
    \item Dynamic analysis: First, dynamic analysis requires to run the application under test. Due
    to ethical reasons, this needs to be done in an isolated environment. Isolating the execution of applications
    by mocking all server communications is a hard problem, especially if the messages are not standardized.
    Second, dynamic analyses require unbounded message probing to produce meaningful results \cite{dtlsimplement}. Sending
    unbounded messages to analyze the payment protocol over NFC interface is infeasible.
    \item Static analysis: To make reverse engineering hard and static analysis infeasible, payment
    application developers use advanced obfuscation techniques. During our initial investigation, we found that
    some of the applications uses code virtualization, which is hard to circumvent \cite{vmhunt}.

\end{itemize}

\textbf{Our Approach: } We use regular inference, also known as automata learning \cite{ANGLUIN198787}, to extract a behavioral 
abstraction of the NFC wallet application. The wallet being tested is called System Under Test (SUT). It takes sequence of 
input messages and responds accordingly by running the payment service. The inference algorithms \cite{Niese2003AnIA} are responsible 
to feed sequence of inputs to SUT and observe the outputs to infer a Mealy machine which is a special form of FSM. We also patch the original 
apk (Android Package) to bypass NFC interface which is an obstacle towards building automated black box analysis 
framework. To learn FSMs from the SUT, we use active FSM learning approach \cite{dikeue} \cite{dynamictesting}. 
Using this approach on contactless android wallets has several protocol-specific challenges. First, an application crash 
may force the output of our queries to be non-deterministic which is a violation of a pre-requirement of applying active automata learning. 
Second, a reset has to be implemented in the wallet so that after each sequence of messages the wallet deletes all 
internal states. We choose a software based reset approach for implementing that. Third, we have to convert the 
abstract symbols into corresponding apdu messages. 